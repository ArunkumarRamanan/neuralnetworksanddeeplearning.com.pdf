% PAGE DIMENSIONS
% FOR A5 without scaling
%\usepackage[inner=18mm, outer=10mm,top=20mm,headsep=10mm,bottom=10mm,paperwidth=148mm,paperheight=210mm]{geometry}
% FOR 17x24cm
\usepackage[inner=30mm, outer=20mm,top=24mm,headsep=10mm,bottom=20mm,paperwidth=170mm,paperheight=240mm]{geometry}
% FOR A4
%\usepackage[inner=50mm, outer=40mm,top=60mm,headsep=10mm,bottom=57mm,paperwidth=210mm,paperheight=297mm]{geometry}
%\usepackage[inner=8.5mm, outer=12.5mm,top=10mm,headsep=5mm,bottom=10mm,paperwidth=141mm,paperheight=200mm]{geometry}
%\usepackage[inner=30mm, outer=23mm,top=29mm,headsep=10mm,bottom=23mm,paperwidth=210mm,paperheight=297mm]{geometry}
% FOR A5
%\usepackage[inner=25mm, outer=17mm,top=8mm,headsep=8mm,bottom=17mm,paperwidth=148mm,paperheight=210mm,includehead]{geometry}
% TABLES: full-width using X{10cm} column
\usepackage{tabularx}
% DRAWINGS (for chapter title)
\usepackage{tikz}
% TYPOGRAPHY: 
\usepackage[protrusion=true,final,factor=1500]{microtype}

% SOME extra math symbols
\usepackage{amssymb}

% TEXT on top of pictures
% ABS or REL positioning
% REL is better for scaled figures
\usepackage[rel]{overpic}
% CHEMICAL formulae like \ce{SiO2} and \ce{(NH4)2S2O8}
\usepackage[version=4]{mhchem}



% CAPTIONS of figures/table styling
\usepackage[font={stretch=1.1,small},labelfont=bf]{caption} % order matters: stretch, then size
%\usepackage[font={small,sf},labelfont=bf]{caption} % small sans-serif + bold

% FONT  SIZES
\usepackage{fix-cm} % FOR ANY font sizes, not predefined
\renewcommand\footnotesize{\fontsize{7pt}{7pt}\selectfont}
\renewcommand\small{\fontsize{6pt}{6pt}\selectfont}
\renewcommand\normalsize{\fontsize{9pt}{9pt}\selectfont}
\renewcommand\large{\fontsize{10pt}{10pt}\selectfont}
\renewcommand\Large{\fontsize{11pt}{11pt}\selectfont}
\renewcommand\Huge{\fontsize{24pt}{24pt}\selectfont}

%\renewcommand\small{\fontsize{10pt}{10pt}\selectfont}
%\renewcommand\normalsize{\fontsize{12pt}{12pt}\selectfont}
%\renewcommand\large{\fontsize{14pt}{14pt}\selectfont}
%\renewcommand\Large{\fontsize{16pt}{16pt}\selectfont}
%\renewcommand\Huge{\fontsize{36pt}{36pt}\selectfont}

% LINE spacing
\renewcommand{\baselinestretch}{1.25}

% FONTS for LaTeX/XeLaTeX
\usepackage{ifxetex}
\ifxetex
			\usepackage{mathspec}
			\usepackage{polyglossia}
			\setdefaultlanguage[variant=uk]{english}
			\defaultfontfeatures{Ligatures=TeX} % To support LaTeX quoting style
			\setmainfont{Minion Pro} 
			\setsansfont{Myriad Pro}
			\setmathsfont(Digits,Greek,Latin)[Numbers={Proportional}]{Minion Pro}
			\setmathrm{Minion Pro}
			\usepackage[italic]{mathastext}
			%\setmainfont[BoldFont={SwiftNeueLTW01-Bold},ItalicFont={SwiftNeueLTW01-Italic}]{SwiftNeueLTW01} 
			%\setsansfont{UniversLTW01-55Roman}
\else
			%\usepackage[T1]{fontenc}
			%\usepackage[utf8]{inputenc}
			\usepackage[english]{babel}
			%\usepackage{fouriernc} % UTOPIA + FOURIER
			%\usepackage{mathpple} % palatino
			%\usepackage[sc]{mathpazo}
			\usepackage[charter]{mathdesign} %utopia, garamond
			\usepackage[scaled]{helvet}
			%\usepackage[light]{roboto}
			%\renewcommand{\sfdefault}{ua1}
\fi


% HEADER/FOOTER
\usepackage{fancyhdr}
\renewcommand{\headrulewidth}{0pt}
\renewcommand{\footrulewidth}{0pt}
\fancyhf{}
\makeatletter
	\let\ps@plain\ps@empty
\makeatother
% PLAIN: for first page of chapter
\fancypagestyle{plain}{
	\renewcommand{\headrulewidth}{0pt}
	\fancyhf{}
	\fancyhead[RO]{
		\makebox[2cm][l]{
			\makebox[4cm][c]{\normalsize
				\hskip0.25em\phantom{\thepage}\phantom{XX}~~$\left|\vphantom{\int_a^b}\right.$~~\thepage\phantom{XX}
			}
		}
	}
	\fancyhead[LE]{
		\makebox[2cm][r]{
			\makebox[4cm][c]{\normalsize
				\hskip0.45em\phantom{XX}\thepage~~$\left|\vphantom{\int_a^b}\right.$~~\phantom{XX}\phantom{\thepage}
			}
		}
	}
}
% REGULAR pages
\fancyhead[RO]{
	\makebox[2cm][l]{
		\makebox[4cm][c]{\normalsize
			\hskip0.25em\phantom{\thepage}\nouppercase\rightmark~~$\left|\vphantom{\int_a^b}\right.$~~\thepage\phantom{\nouppercase\rightmark}
		}
	}
}
\fancyhead[LE]{
	\makebox[2cm][r]{
		\makebox[4cm][c]{\normalsize
			\hskip0.45em\phantom{\nouppercase\leftmark}\thepage~~$\left|\vphantom{\int_a^b}\right.$~~\nouppercase\leftmark\phantom{\thepage}
		}
	}
}

% MARGIN labels, except chapter=0 (intro; conclusion; bibliography etc.)
% AND \setcouter{chapter}{0} has to be set explicitely in required chapters
\fancyfoot[RO]{
	\ifnum\value{chapter}>0
	\begin{tikzpicture}[remember picture, overlay]
	\node[rounded corners=2mm,inner sep=3mm,anchor=north east,black,fill=black!15,draw=black!75] at ([xshift=2mm,yshift=-\arabic{chapter}*1.3cm-1.1cm]current page.north east) {\fontsize{1cm}{1cm}\selectfont\thechapter};
	\end{tikzpicture}
	\fi
}
\fancyfoot[LE]{
	\ifnum\value{chapter}>0
	\begin{tikzpicture}[remember picture, overlay]
	\node[rounded corners=2mm,inner sep=3mm,anchor=north west,black,fill=black!15,draw=black!75] at ([xshift=-2mm,yshift=-\arabic{chapter}*1.3cm-1.1cm]current page.north west) {\fontsize{1cm}{1cm}\selectfont\thechapter};
	\end{tikzpicture}
	\fi
}



\pagestyle{fancy}
% HEADER contents - chapter name and section name
%\renewcommand{\chaptermark}[1]{\markboth{\thechapter.\, #1}{}}
\renewcommand{\chaptermark}[1]{\markboth{#1}{}}
\renewcommand{\sectionmark}[1]{\markright{\thesection.\, #1}}


% FLOATING objects
\usepackage{float}
% PICTURES
\usepackage{graphicx}
% NOT USED
\usepackage{setspace}

% SPACINGS in lists
\usepackage{enumitem}
% VERICAL: topsep partopsep parsep itemsep
% HORIZONTAL: leftmargin rightmargin listparindent labelwidth labelsep itemindent
% GLOBAL: \setlist[enumerate]{labelsep=*, leftmargin=1.5pc}
\setlist{noitemsep}
%\setlist{nosep}




%% BIBLATEX
%% BIBliography: if authors <= MAXBIBNAMES - show all, ELSE: show only MINBIBNAMES et al.
%% TEXTCITE: if authors <= MAXCITENAMES - show all, ELSE: show only MINCITENAMES et al.
%% SORTCITES = sort numbers in ascending order
%% IGNORE/NOT IGNORE URL, DOI, EPRINT fields
%% BIBstyle = ieee (available)/ nature / science etc. (requires *.bbx file)
%% NICE: ieee+doi=true and phys+doi=false
%\usepackage[maxbibnames=3,minbibnames=1,backend=bibtex,maxcitenames=2,mincitenames=1,bibstyle=ieee,citestyle=numeric-comp,sorting=none,sortcites,url=false,doi=true,eprint=false]{biblatex}
%% REDUCE font size for bibliography (=useless waste of space)
%%\renewcommand*{\bibfont}{\footnotesize} %too small
%\renewcommand*{\bibfont}{\small}
%% SUPPRESS 'in' nefore journal name
%\renewbibmacro{in:}{}
%% LIST of bib files
%\addbibresource{library.bib}
%% STYLE of URL/DOI: Serif font instead of monospaced (mono looks ugly)
%\urlstyle{rm}
%
%% REdefine fullcite for personal publications list
%\DeclareCiteCommand{\fullcite}
%{\usebibmacro{prenote}}
%{\usedriver
%	{\defcounter{minnames}{6}%
%		\defcounter{maxnames}{9}}
%	{\thefield{entrytype}}.}
%{\multicitedelim}
%{\usebibmacro{postnote}}
%
%% FOOTnote on the chapter first page without footnotemark
%% RESULTS of this chapter were published as a paper
%\makeatletter
%    \def\blfootnote{\gdef\@thefnmark{}\@footnotetext}
%\makeatother


% TABLES: nicer rulers
% AND nicer spacing between lines in tables
\usepackage{booktabs}
\renewcommand{\arraystretch}{1.2}


% TITLES styling
\usepackage[toctitles,explicit,raggedright]{titlesec}
\newcommand*\chapterlabel{}
% CHAPTER in frontmatter|backmatter
\titleformat{name=\chapter,numberless}[display]
	{\normalfont\rmfamily\Huge\bfseries}{}{1ex}
	{\flushright{\chapterlabel#1}}
% CHAPTER in mainmatter
\titleformat{\chapter}
{\gdef\chapterlabel{}\normalfont\rmfamily\Huge\bfseries}
{\gdef\chapterlabel{}}{-10em}
{
	\flushright{
		\begin{tikzpicture}
			%\draw[help lines,step=5mm] (0,-3) grid (-\linewidth,3);
			\node[black!50,anchor=east,inner sep=0mm] (a) at (0,0) {\fontsize{7cm}{8cm}\selectfont\thechapter};
			\begin{scope}[cm={1,0,-0.6,0.15,(0,0)}].
					\node[transform shape,black!30,anchor=south,inner sep=0mm] at (a.south) {\fontsize{7cm}{8cm}\selectfont{}\thechapter};
					%\node[transform shape,black!30,anchor=south east,inner sep=0mm] at (a.south) {\fontsize{3cm}{8cm}\selectfont{}\chaptername};
			\end{scope}
			\node[black!50,anchor=east,inner sep=0mm] (a) at (0,0) {\fontsize{7cm}{8cm}\selectfont\thechapter};
			\node[black!35,anchor=east,inner sep=0.20mm,scale=0.98] at (0,0) {\fontsize{7cm}{8cm}\selectfont\thechapter};
			\node[black!20,anchor=east,inner sep=0.40mm,scale=0.96] at (0,0) {\fontsize{7cm}{8cm}\selectfont\thechapter};
			%\node[black!20,scale=0.97] at (a) {\fontsize{6cm}{8cm}\selectfont\thechapter};
			%\node[anchor=east,black!45] at (0,-1) {\resizebox{\linewidth}{!}{\chaptername}};
			%\node[anchor=north east,inner sep=0mm] at (a.north east) {\parbox{\linewidth}{\raggedleft\chapterlabel#1}};
			\node[anchor=east,inner sep=0mm] at (0,0) {\parbox{\linewidth}{\raggedleft\chapterlabel#1}};
		\end{tikzpicture}
		%\chapterlabel#1
	}
}
% SPACING: chapter by default uses \@makechapterhead with extra spacing before and after the chapter title
\titlespacing*{\chapter}{0pt}{-25pt}{30pt}

%\titleformat{\section}[block]{\Large}{\bfseries\thesection.\,\,#1}{1em}{}
%\titleformat{\subsection}[block]{\large}{\bfseries\thesubsection.\,\,#1}{1em}{}
%\titleformat{\subsection}{\no}{\itshape\thesubsection.\,#1}{1em}{}


% GREEK letters in section/chapter titles AND in PDF bookmarks
%\usepackage[artemisia]{textgreek}


% PART page picture
% DEFINE \partimage before the beginning of the part
\makeatletter
\newcommand{\partimage}[2][]{\gdef\@partimage{\includegraphics[#1]{#2}}}
% REDEFINED amsbook part code to include the \@partimage insertion command - copypaste from SeX
\def\@part[#1]#2{%
	\ifnum \c@secnumdepth >-2\relax \refstepcounter{part}%
	\addcontentsline{toc}{part}{\partname\ \thepart.
		\protect\enspace\protect\noindent#1}%
	\else
	\addcontentsline{toc}{part}{#1}\fi
	\begingroup\centering
	\ifnum \c@secnumdepth >-2\relax
	{\fontsize{\@xviipt}{22}\bfseries
		\partname\ \thepart} \vskip 20\p@ \fi
	\fontsize{\@xxpt}{25}\bfseries
	#1\vfil\@partimage\vfil\endgroup \newpage\thispagestyle{empty}}
\makeatother


\usepackage{listings}
\definecolor{mygreen}{rgb}{0,0.6,0}
\definecolor{mygray}{rgb}{0.5,0.5,0.5}
\definecolor{mymauve}{rgb}{0.58,0,0.82}
\usepackage{newtxtt}
\usepackage{amsmath}
\usepackage{pgfplots}

\lstset{ 
	backgroundcolor=\color{white},   % choose the background color; you must add \usepackage{color} or \usepackage{xcolor}; should come as last argument
	basicstyle=\small\ttfamily,
	belowcaptionskip=0em,        % the size of the fonts that are used for the code
	belowskip=-2em,
	breakatwhitespace=false,         % sets if automatic breaks should only happen at whitespace
	breaklines=true,                 % sets automatic line breaking
	captionpos=b,                    % sets the caption-position to bottom
	commentstyle=\color{mygreen},    % comment style
	deletekeywords={...},            % if you want to delete keywords from the given language
	escapeinside={\%*}{*)},          % if you want to add LaTeX within your code
	extendedchars=true,              % lets you use non-ASCII characters; for 8-bits encodings only, does not work with UTF-8
	frame=single,	                   % adds a frame around the code
	keepspaces=true,                 % keeps spaces in text, useful for keeping indentation of code (possibly needs columns=flexible)
	keywordstyle=\color{blue},       % keyword style
	language=Python,                 % the language of the code
	morekeywords={*,...},            % if you want to add more keywords to the set
	numbers=none,                    % where to put the line-numbers; possible values are (none, left, right)
	numbersep=5pt,                   % how far the line-numbers are from the code
	numberstyle=\tiny\color{mygray}, % the style that is used for the line-numbers
	rulecolor=\color{black},         % if not set, the frame-color may be changed on line-breaks within not-black text (e.g. comments (green here))
	showspaces=false,                % show spaces everywhere adding particular underscores; it overrides 'showstringspaces'
	showstringspaces=false,          % underline spaces within strings only
	showtabs=false,                  % show tabs within strings adding particular underscores
	stepnumber=2,                    % the step between two line-numbers. If it's 1, each line will be numbered
	stringstyle=\color{mymauve},     % string literal style
	tabsize=2,	                   % sets default tabsize to 2 spaces
	title=\lstname 			          % show the filename of files included with \lstinputlisting; also try caption instead of title
}
